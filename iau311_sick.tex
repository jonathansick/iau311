\documentclass{iau}
\usepackage{graphicx,natbib,url}
\bibliographystyle{apj}
 
\newcommand{\apj}{ApJ}           % Astrophysical Journal
\newcommand{\apjl}{ApJ}           % Astrophysical Journal
\newcommand{\mnras}{MNRAS}       % Monthly Notices of the RAS
\newcommand{\nat}{Nature}
\newcommand{\aap}{A\&A}
\newcommand{\aaps}{A\&AS}
\newcommand{\araa}{ARA\&A}
\newcommand{\aj}{AJ}
\newcommand{\pasp}{PASP}
\newcommand{\apjs}{ApJS}           % Astrophysical Journal
\newcommand{\aapr}{A\&A Rev.}
\newcommand{\na}{New Astr.}

\title{The Stellar Mass of M31 seen by the Andromeda Optical \& Infrared Disk Survey}

\author[Sick et al]{Jonathan Sick,$^1$  Stephane Courteau,$^1$ Jean-Charles Cuillandre,$^2$ Julianne Dalcanton,$^3$ Roelof de Jong,$^4$ Michael McDonald,$^5$ Dana Simard,$^1$ \and R. Brent Tully$^6$}

\affiliation{$^1$Department of Physics, Engineering Physics \& Astronomy, Queen's University, Kingston, ON, Canada K7L 3N6. email: {\tt jsick@astro.queensu.ca}, {\tt courteau@astro.queensu.ca}\\
$^2$CEA IRFU\\
$^3$Department of Astronomy, University of Washington, Box 351580, Seattle, WA 98195, USA. {\tt jd@astro.washingston.edu}\\
$^4$Leibniz Institut für Astrophysik Potsdam (AIP), An der Sternwarte 16, 14482 Potsdam, Germany. {\tt rdejong@aip.de}\\
$^6$Kavli Institute for Astrophysics and Space Research, MIT, Cambridge, MA, USA. {\tt mcdonald@space.mit.edu}\\
$^6$Institute for Astronomy, University of Hawaii, 2680 Woodlawn Drive, Honolulu, HI, USA. {\tt tully@ifa.hawaii.edu}}

\pubyear{2014}
\volume{311}
\jname{Galaxy Masses as Constraints of Formation Models}
\editors{M. Cappellari \& S. Courteau, eds.}

% Figure template
% \begin{figure}
% \centering
% \includegraphics[width=0.5\columnwidth]{cappellari_fig1.eps} 
% \caption{Lorem ipsum dolor sit amet, consectetur adipiscing elit. Duis vel erat eget orci auctor vestibulum fermentum at nibh. Sed eget tempor velit, ac mattis tortor. Integer eget tincidunt dolor. Pellentesque et dui vitae arcu vehicula blandit quis scelerisque velit. Maecenas ultricies lacus ac nibh dapibus tincidunt. Aenean consequat faucibus magna, at malesuada arcu scelerisque vitae.}\label{fig:fig1}
% \end{figure}

\begin{document}

\maketitle

\begin{abstract}
Our proximity and external vantage point make M31 an ideal testbed for understanding the structure of spiral galaxies.
The Andromeda Optical and Infrared Disk Survey (ANDROIDS) has now mapped M31's bulge and disk out to R=40 kpc in $ugriJK_s$ bands with CFHT using a careful sky calibration programme.
We use Bayesian modelling of the optical-infrared spectral energy distribution to estimate profiles of M31's stellar populations and mass along the major axis.
This analysis provides evidence for inside-out disk formation and a declining metallicity gradient.
M31's $i$-band mass-to-light ratio ($M/L_i^*$) decreases from 0.5~dex in the bulge to $\sim 0.2$~dex at 40~kpc.
The best-constrained models are made against the full $ugriJK_s$ SED, and are also consistent with optical-only SED fits.
Thus NIR data can be successfully modelled with modern stellar population synthesis, though NIR data do not appear to provide significant extra constraints in this application.
Fits to the $gi$-SED alone yield $M/L_i^*$ that are systematically lower by 0.1~dex.
Relations of $M/L_i$ via $g-i$ colour found in the literature systematically differ by 0.3~dex, reminding us of the importance of assumptions built into stellar population models in estimating accurate stellar masses.
We advocate a stellar mass of $M_*(30~\mathrm{kpc})=10.3^{+2.3}_{-1.7}~\mathrm{M}_\odot$.
\keywords{galaxies: elliptical and lenticular, cD - galaxies: evolution - galaxies: formation}
\end{abstract}

\firstsection
\section{Introduction}

ANDROIDS is a programme with the MegaCam and WIRCam cameras on the Canada-France-Hawaii Telescope (CFHT) that has homogeneously mapped M31's bulge and disk within $R=40$~kpc with $ugriJK_s$ bands to enable global studies of M31's structure and stellar populations using both resolved stars and integrated spectral energy distributions (SEDs).
In this contribution, we use ANDROIDS to estimate the stellar mass profile of the M31 disk with Bayesian modelling of the optical to near-IR (NIR) SED.
This approach is more rigorous than the colour-$M/L^*$ prescriptions \citep[e.g.][]{Zibetti:2009,Taylor:2011,Into:2013} often employed by pixel-by-pixel stellar mass estimation studies that use only a $g-i$ colour and marginalize over all likely star formation histories.
By studying M31 in detail, an overall goal of ANDROIDS is to explore systematic uncertainties in studies of more distant and poorly resolved systems.

\section{M31 Surface Brightness Calibration}

Background subtraction is the most significant challenge for observational studies of M31's structure since we cannot observe the M31 disk and blank sky in the same fields of view.
This is particularly acute in our NIR maps where skyglow is 3-dex brighter than the disk, while also have strong spatial and temporal variations.
In \cite{Sick:2014} we described our ANDROIDS/WIRCam sky-target nodding and background subtraction schemes and find that the NIR background cannot be known to better than 2\% given the scale of sky-target nods required for M31.
We can overcome this uncertainty by solving for sky offsets that formally minimize surface brightness differences between overlapping pairs of images.
Such sky offsets are $\sim1$\% of the NIR brightness, but systematically uncertain up to a zeropoint normalization of 0.16\% of the sky level.
In optical bands the sky background is both more stable and far less bright, though we still employ sky-target nodding with the Elixir-LSB method for CFHT/MegaCam to build a real-time map of sky and scattered light backgrounds over one-hour sliding windows.
With Elixir-LSB we easily identify low surface brightness features in M31's outer disk, such as the Norther Spur, at levels below $\mu_g\sim26$~mag~arcsec$^{-2}$.

The aforementioned sky offset zeropoint uncertainty requires that our surface brightness profiles be finely calibrated against external datasets.
Resolved stellar catalogs transformed into surface brightness maps, such as our own WIRCam star catalog, but even Panchromatic Hubble Andromeda Treasury \citep[PHAT;][]{Dalcanton:2012} provide a useful dataset up to the limit of completeness corrections.
Extremely wide-field imaging is also useful since the background can be sampled simultaneously with the disk light.
We are currently using Dragonfly \citep{Abraham:2014} to image M31 and replace the wide-field Schmidt telescope plates of \cite{Walterbos:1987}.

\section{SED Stellar Mass Modelling}

\begin{figure}
\centering
\includegraphics[width=0.7\columnwidth]{pop_profile} 
\caption{Posterior stellar population profiles for different bandpass sets: $ugriJK_s$ (blue), $ugri$ (green), $gi$ (red).
A declining metallicity gradient and inside-out disk formation (seen by an increase in the e-folding time ($\log \tau$) of the exponentially declining star formation history model) are clearly evident.}
\label{fig:pop_profile}
\end{figure}

\begin{figure}
\centering
\includegraphics[width=0.7\columnwidth]{mass_profile} 
\caption{Posterior stellar $M/L_i^*$ (left) and stellar mass (right) profiles.,
The $ugriJK_s$ (blue) and $ugri$ (green) fits are consistent, while MCMC chains modelling only $gi$ are lower by 0.1~dex in $M/L_i^*$.
Equivalent $gi$--$M/L_i^*$ relations in the literature can vary by 0.3~dex of $M/L_i^*$.}
\label{fig:mass_profile}
\end{figure}

From the calibrated surface brightness profiles, we model the SED at each radial bin to estimate the stellar population, and hence the stellar mass-to-light ratio, $M/L_i^*$.
Our modelling engine is the Flexible Stellar Population Synthesis (FSPS) software \citep{Conroy:2009,Conroy:2010}.
We chose FSPS both because it is well-calibrated and known to have a `lighter' AGB contribution than older SP models \citep[e.g.,][]{Bruzual:2003} and also because it's application programming interface allows deep customization of the computed stellar populations.
One author (J.S.) contributes to the maintenance of a Python-language wrapper for FSPS.\footnote{\url{http://dan.iel.fm/python-fsps}}
We use a Markov Chain Monte Carlo approach to modelling SEDs extracted along the northern major axis of the M31 disk implemented with the \texttt{emcee} python package \citep{Foreman-Mackey:2013}.
We tested different star formation history parameterizations and found that a simple `$\tau$' model, involving constant plus exponentially declining star formation rate components minimized residuals compared to more sophisticated `delayed $\tau$' and late burst models.
Of the dust attenuation treatments, the default power-law attenuation law with separate components for young and older stellar populations also minimized residuals compared to Milky Way or starburst attenuation models.

We found that posterior SED residuals are minimized by fitting the entire $ugriJK_s$ SED.
This contrary to previous advice from \cite{Taylor:2011} and \cite{Zibetti:2009} who advocated ignoring NIR bands in mass estimation due to uncertain AGB treatments of the previous generation of stellar population synthesis models \citep[e.g.][]{Bruzual:2003,Maraston:2005}.
Besides the NIR, the $u$-band is also crucial: the $griJK_s$-fit SED had has little predictive power over the $u$-band as the optical fit had over the NIR.
This result should encourage SED modellers to incorporate as many bandpasses as possible, including UV and IR, to obtain the best constraints on stellar populations and masses.

We modelled SEDs extracted from a logarithmically-sized wedge \citep[e.g.][their Fig. 2]{Courteau:2011} to produce stellar population profiles (shown in Fig. \ref{fig:pop_profile}).
Interestingly, the $ugri$-fit and $ugriJK_s$-fit SEDs produce statistically identical stellar population profiles, with the only exception being a slightly tighter posterior credible region from the full-SED fits.
Although the consistency of optical and optical-NIR SED fits is reassuring from the perspective of NIR calibrations, it is also disappointing that the NIR data has not produced a remarkably improved posterior stellar population estimate.
% Certainly this result is anecdotal, and could change with different underlying stellar populations and dust distributions.

What is clear, though, is that poorly sampled SEDs can bias results.
Fitting only the $gi$ SED (that is, using an input information equivalent to those using colour-$M/L^*$ look-up-tables) clearly biases the posterior stellar population distribution, with significantly lower dust opacities and lower mass-to-light ratios. 
By comparison, we have also plotted mass-to-light ratios predicted by three colour-$M/L^*$ relations \citep{Zibetti:2009,Taylor:2011,Into:2013}.
These fits systematically vary by 0.3~dex, far larger than the 0.1~dex of internal systematic uncertainty typically claimed by $g-i$ -- $M/L$ fits. 
Compared to our full SED fits, modelling of the $gi$ SED is less biased than these other $M/L^*$ fits, which are based on other stellar population synthesis models.
This serves as reminder that stellar mass estimates remain dominated by prior assumptions such as choices of IMF, dust, and details of AGB treatments, among other concerns.

Given this discussion, we estimate a stellar mass of the M31 disk, within $30$~kpc, as $M_{ugri}^{*} = 10.3^{+2.3}_{-1.7}~\mathrm{M}_\odot$.
This result is consistent with the stellar bulge and disk masses quoted by \cite{Tamm:2012} ($10.1\times10^{10}~\mathrm{M}_\odot$).

\section{Discussion}

We have used $ugriJK_s$ SEDs to map the stellar mass of M31's disk.
We are currently using these stellar mass maps in concert with dynamical tracers of gas and stars to construct a mass model of M31's stellar, gas, and dark matter components.
% \cite{Tamm:2012} recently constructed a mass model of M31 using dynamical tracers along with SDSS and Spitzer 3.6~$\mu$m photometry.
% Their bulge models are ill-constrained due to a lack of dynamical tracers within 8~kpc.
The DiskFit code \citep{Kuzio-de-Naray:2012} allows us to correct for non-circular motions in HI kinematical measurements of \cite{Saglia:2010} and \cite{Chemin:2009} that reach the centre of M31.
The success of this mass model will be determined by the stability of a dynamical N-body realization.

\section*{Acknowledgements}

\noindent J.S. and S.C. acknowledge support through respective Graduate Scholarship and Discovery grants from the Natural Sciences and Engineering Research Council of Canada.
We thank the Canadian Advanced Network for Astronomical Research (CANFAR) for furnishing the computing facilities for this work.

% \begin{thebibliography}{99}

% \bibitem[{{Bruzual} \& {Charlot}(2003)}]{bruzual03}
% {Bruzual} G., {Charlot} S., 2003, \mnras, 344, 1000

% \end{thebibliography}

\bibliography{master}

\end{document}
